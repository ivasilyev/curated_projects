\documentclass[12pt,a4paper]{article}
\usepackage[utf8]{inputenc}
\usepackage[T1]{fontenc} 
\usepackage[english]{babel}
\usepackage{csquotes}
\usepackage[hidelinks]{hyperref}
\nonfrenchspacing
\usepackage{pgfplotstable}  % TSV data support
\usepackage[backend=biber, % style=authoryear,
natbib=true]{biblatex}  % Bibliography support
\usepackage[acronym]{glossaries}  % Abbreviations & species support
\addbibresource{latexrefs.bib}

% Define species
\newacronym[first={\textit{Escherichia coli}}]{ecoli}{\textit{E. coli}}{\textit{Escherichia coli}}
\newacronym[first={\textit{Klebsiella pneumoniae}}]{kpne}{\textit{K. pneumoniae}}{\textit{Klebsiella pneumoniae}}
% Define abbreviations
\newacronym{mdrkp}{MDRKP}{Multi drug resistant \gls{kpne}}

\title{Multidrug-resistant Klebsiella pneumoniae found in infant gut microbiota}
\date{\today}
\author{Vasilyev IY}

\begin{document}
\maketitle

\begin{abstract}
TODO
\end{abstract}

\section{Introduction}\label{sec:intro}
Newborn patients could obtain microorganisms from clinical environment, personnel, other patients and parents,
e.g.\ via breast milk.
Since Enterobacteriaceae are known early human gastrointestinal tract colonisers, local and systemic diseases could take
place during the dynamic microbiome development.
Their persistence as stable colonizers may have destructive consequence on the host vital functions.
Moreover, due to the high horizontal gene transport frequency between the microbiome parts, persistence of even single
strain carrying pathobiotic genes after its spread may cause explicit cyto- and genotoxic effects on host cells
leading to dangerous repercussion including colorectal cancer in particular~\cite{Pope2019}.
Underweight and weakened patients of neonatal intensive care nurseries often suffer Gram-positive bacteria infections,
mostly caused by coagulase-negative Staphylococci, while Gram-negative bacteria infections are considered as more rare
and deadly after even more rapid generalization~\cite{Dorota2017}.

\gls{kpne} is the causative agent of numerous nosocomial and community acquired infections including
pneumonia, sepsis, bacteremia, meningitis, pyogenic liver abscesses, urinary tract infections and more.
The risk group historically includes patients with weakened and malfunctioning immune system, but the spread of
hypervirulent strains also endangers immunosufficient individuals~\cite{Shankar2018}.
First isolated from lungs of patient with pneumonia \textit{postmortem}, \gls{kpne} were acknowledged as a part of
normal human gastrointestinal tract microbiome since then.
The colonization can spread, persist for years and cause different pathologies from hidden carriage to
fatal acute infections even for the healthy individuals~\cite{Martin2018}.

In this article we present 8 cases of mature infants with the no-manifested carriage of multi drug resistant
\gls{kpne} (\gls{mdrkp}) in the gastrointestinal tract detected at some time after the birth.
The source(s) of the \gls{mdrkp} infection was not defined.

\section{Materials and methods}\label{sec:mat_met}
\subsection{Isolation of rectal \gls{mdrkp} samples from the infant patients}\label{subsec:iso}
A total of 10 \gls{kpne} isolates demonstrating multidrug resistance phenotype were collected from 10
stool samples of 8 monitored newborn patients during hospitalization in maternity hospital of Kazan, Russia.
5 infants were born with vaginal delivery, 3~--- after caesarean surgery.
No microbial DNA was found in the meconium samples from the all 8 individuals.
However, the stool samples collected at 4~--- 5 day of life were contaminated with $10^8$..$10^9$ of \gls{mdrkp}
colony-forming units per gram of stool.
Two samples collected from two infants after 1 and 6 months of monitoring also contained \gls{mdrkp}.
No manifested symptoms were reported across the whole monitoring time from all patients.

\subsection{Phenotypic Characterization}\label{subsec:phe}
The antimicrobial susceptibilities of the isolates were determined using a broth microdilution procedure.
The results were interpreted in automated mode using VITEK 2 Compact analyzer (bioMérieux SA, France) according to
producer's guidance documents.

\subsection{Whole-genome sequencing and assembly.}\label{subsec:proc_raw}
Libraries were prepared using Nextera XT DNA Library Preparation Kit.
Whole-genome DNA was sequenced using Illumina MiSeq platform (Illumina Inc., USA),
with a paired-end run of 2 by 250 bp.
Raw reads quality control was performed with FastQC v0.11,  % quay.io/biocontainers/fastqc:0.11.8--1
% fastqc -t 32 sample.fastq.gz -o sample
trimmed by trimmomatic v0.39  % quay.io/biocontainers/trimmomatic:0.39--1
% trimmomatic PE -threads 32 -phred33 sample.1.fastq.gz sample.2.fastq.gz sample_trimmomatic.1.fastq.gz sample_trimmomatic_untrimmed.1.fastq.gz sample_trimmomatic.2.fastq.gz sample_trimmomatic_untrimmed.2.fastq.gz ILLUMINACLIP:adapters.fasta:2:30:10 LEADING:3 TRAILING:3 SLIDINGWINDOW:4:15 MINLEN:36
and cutadapt v2.4,  % quay.io/biocontainers/cutadapt:2.4--py37h14c3975_0
% cutadapt -a AGATCGGAAGAG -A AGATCGGAAGAG -m 50 -o sample_cutadapt.1.fastq.gz -p sample_cutadapt.2.fastq.gz sample_trimmomatic.1.fastq.gz sample_trimmomatic.2.fastq.gz
assembled using SPAdes v3.9.1.  % quay.io/biocontainers/spades:3.9.1--0
% spades --careful -o sample/genome -1 sample_cutadapt.1.fastq.gz -2 sample_cutadapt.2.fastq.gz
% spades --careful -o sample/plasmid -1 sample_cutadapt.1.fastq.gz -2 sample_cutadapt.2.fastq.gz --plasmid

\subsection{Further genome assembly processing.}\label{subsec:proc_ass}
For each sample genome and plasmid assemblies were merged, filtered and deduplicated using own scripts.
The resulting assemblies were submitted to NCBI.\
The corresponding Whole Genome Shotgun project has been deposited at NCBI GenBank database under the main
BioProject accession ID PRJNA556398.
The assemblies were annotated locally with prokka v1.13.7,   % quay.io/biocontainers/prokka:1.13.7--pl526_0
% prokka --compliant --centre UoN --cpu 32 --outdir prokka/sample --force --prefix sample --locustag sample --genus Klebsiella --species pneumoniae sample.fasta
MLST results were computed using SRST2 v0.2.  % quay.io/biocontainers/srst2:0.2.0--py27_2
% getmlst.py --species "Klebsiella pneumoniae"
% srst2 --output sample --input_pe sample_cutadapt.1.fastq.gz sample_cutadapt.2.fastq.gz --mlst_db Klebsiella_pneumoniae.fasta --mlst_definitions kpneumoniae.txt --mlst_delimiter '_' --log --threads 32

\section{Results and discussion}\label{sec:res_dis}
TODO
%\pgfplotstabletypeset[
%  string type,
%]{../../datasets/antibiogram.tsv}
\section{Acknowledgments}\label{sec:acks}
TODO

\printbibliography
Compiled from \LaTeXe{}.
\end{document}
