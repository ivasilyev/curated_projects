\documentclass[12pt,a4paper]{article}
\usepackage[utf8]{inputenc}
\usepackage[T1]{fontenc} 
\usepackage[english]{babel}
\usepackage{csquotes}
\usepackage[hidelinks]{hyperref}
\usepackage{authblk}  % Add the affiliation to the author's name
\usepackage{pgfplotstable}  % TSV data support
\usepackage{graphicx}  % Table rotate support
\usepackage{mfirstuc}  % For First Letter Capitalization
\usepackage{booktabs}
\usepackage[backend=biber, % style=authoryear,
natbib=true]{biblatex}  % Bibliography support
\usepackage[acronym]{glossaries}  % Abbreviations & species support
\addbibresource{latexrefs.bib}
\nonfrenchspacing

% Define macros
\newcommand{\betalactam}{$\beta$-lactam}

% Define species
% \newacronym{<label>}{<abbrv>}{<full>}
\newacronym[first={\textit{Klebsiella pneumoniae}}]{kpne}{\textit{K. pneumoniae}}{\textit{Klebsiella pneumoniae}}
% Define abbreviations
\newacronym{esbl}{ESBL}{extended spectrum \betalactam ase}
\newacronym{mlst}{MLST}{\textit{in silico} multi-locus sequence type}
\newacronym{mdrkp}{MDRKP}{multi drug resistant \gls{kpne}}
\newacronym{wgs}{WGS}{whole genome shotgun}


% Define authors
\author[1]{Vasilyev, I. Y.} % Corresponding author, e-mail: u0412u0418u042e@gmail.com
\author[2]{Nikolaeva, I. V.}  % irinanicolaeva@mail.ru
\author[1]{Siniagina, M. N.}  % marias25@mail.ru
\author[1]{Kharchenko, A. M.}  % anastasiahm@list.ru
\author[2]{Shaikhieva, G. S.}  % studentgulya@yandex.ru
% Define affilations
\affil[1]{Institute of Fundamental Medicine and Biology, Kazan Federal University, Kazan, Russia}
\affil[2]{Kazan State Medical University, Kazan, Russia}

\title{Multidrug-resistant hypervirulent \Gls{kpne} found persisting silently in infant gut microbiota}
\date{\today}

\begin{document}
\maketitle
\glsresetall

\begin{abstract}
TODO
\end{abstract}

\section{Introduction}\label{sec:intro}
Newborn patients could obtain microorganisms from clinical environment, personnel, other patients and parents,
e.g.\ via breast milk.
Since Enterobacteriaceae are known early human gastrointestinal tract colonisers, local and systemic diseases could take
place during the dynamic microbiome development.
Their persistence as stable colonizers may have destructive consequence on the host vital functions.
Moreover, due to the high horizontal gene transport frequency between the microbiome parts, persistence of even single
strain carrying pathobiotic genes after its spread may cause explicit cyto- and genotoxic effects on host cells
leading to dangerous repercussion including colorectal cancer in particular~\cite{Pope2019}.
Underweight and weakened patients of neonatal intensive care nurseries often suffer Gram-positive bacteria infections,
mostly caused by coagulase-negative Staphylococci, while Gram-negative bacteria infections are considered as more rare
and deadly after even more rapid generalization~\cite{Dorota2017}.

\gls{kpne} is the causative agent of numerous nosocomial and community acquired infections including
pneumonia, sepsis, bacteremia, meningitis, pyogenic liver abscesses, urinary tract infections and more.
The risk group historically includes patients with weakened and malfunctioning immune system, but the spread of
hypervirulent strains also endangers immunosufficient individuals~\cite{Shankar2018}.
First isolated from lungs of patient with pneumonia \textit{postmortem}, \gls{kpne} were acknowledged as a part of
normal human gastrointestinal tract microbiome since then.
The colonization can spread, persist for years and cause different pathologies from hidden carriage to
fatal acute infections even for the healthy individuals~\cite{Martin2018}.

In this study we present 8 cases of mature infants with the no-manifested carriage of
\gls{mdrkp} in the gastrointestinal tract detected at some time after the birth.
The source(s) of the \gls{mdrkp} infection was not defined.

\section{Materials and methods}\label{sec:mat_met}
\subsection{Isolation of rectal \gls{mdrkp} samples from the infant patients}\label{subsec:iso}
A total of 10 \gls{kpne} isolates demonstrating multidrug resistance phenotype were collected from 10
stool samples of 8 newborn mature patients during hospitalization in maternity hospital of Kazan, Russia.
5 infants were born with vaginal delivery, 3~--- after caesarean surgery.
No microbiological analysis was performed over parents.
The stool samples collected at 3~--- 4 day of life were contaminated with $10^8$..$10^9$ of \gls{mdrkp}
colony-forming units per gram of stool.
The meconium samples obtained from the all 8 individuals were not tested for contamination.
Yet, the meconium samples obtained from 40 infants during the previous research were sterile or
contained lactobacteria~\cite{Nikolaeva2019a}.
Two samples collected from two infants after 1 and 6 months of monitoring also contained \gls{mdrkp}.
All the 8 babies were discharged from hospital at 4~--- 5 day of life in satisfactory condition.
Blood in stool and liquid stool were reported once for patient \#1 to the third month of life,
and constipation was reported for patient \#2 to the first month of life.
No other manifested and prolonged symptoms were reported across the whole monitoring time from the all individuals.

\subsection{Phenotypic Characterization}\label{subsec:phe}
The antimicrobial susceptibilities of the 10 microbial isolates were determined using a broth microdilution procedure.
The following antibacterial agents were tested: aminoglycosides (amikacin, netilmicin, gentamicin),
\betalactam s (amoxicillin-clavulanic acid, ampicillin, aztreonam, ceftriaxone, imipenem, meropenem),
nitrofuran derivatives (nitrofurantoin), sulfonamides (sulfamethoxazole), 2,4-diaminopyrimidines (trimethoprim),
fluoroquinolones (ciprofloxacin), chloramphenicol, fosfomycin.
The production of \gls{esbl} and susceptibility to \textit{Klebsiella} phage and pyo bacteriophage
were also analyzed during the routine.
The results were interpreted in automated mode using VITEK 2 Compact analyzer (bioMérieux SA, France) according to
producer's guidance documents.

\subsection{Whole-genome sequencing and assembly.}\label{subsec:proc_raw}
Libraries were prepared using Nextera XT DNA Library Preparation Kit.
Whole-genome DNA was sequenced using Illumina MiSeq platform (Illumina Inc., USA),
with a paired-end run of 2 by 250 bp.
Raw reads quality control was performed with FastQC v0.11~\cite{FastQC},  % quay.io/biocontainers/fastqc:0.11.8--1
% fastqc -t 32 sample.fastq.gz -o sample
trimmed by Trimmomatic v0.39~\cite{Trimmomatic}
% quay.io/biocontainers/trimmomatic:0.39--1
% trimmomatic PE -threads 32 -phred33 sample.1.fastq.gz sample.2.fastq.gz sample_trimmomatic.1.fastq.gz sample_trimmomatic_untrimmed.1.fastq.gz sample_trimmomatic.2.fastq.gz sample_trimmomatic_untrimmed.2.fastq.gz ILLUMINACLIP:adapters.fasta:2:30:10 LEADING:3 TRAILING:3 SLIDINGWINDOW:4:15 MINLEN:36
and Cutadapt v2.4~\cite{Cutadapt},
% quay.io/biocontainers/cutadapt:2.4--py37h14c3975_0
% cutadapt -a AGATCGGAAGAG -A AGATCGGAAGAG -m 50 -o sample_cutadapt.1.fastq.gz -p sample_cutadapt.2.fastq.gz sample_trimmomatic.1.fastq.gz sample_trimmomatic.2.fastq.gz
assembled using SPAdes v3.9.1~\cite{SPAdes}.
% quay.io/biocontainers/spades:3.9.1--0
% spades --careful -o sample/genome -1 sample_cutadapt.1.fastq.gz -2 sample_cutadapt.2.fastq.gz
% spades --careful -o sample/plasmid -1 sample_cutadapt.1.fastq.gz -2 sample_cutadapt.2.fastq.gz --plasmid
Assembly statistics were calculated using the reference \gls{kpne} genome with RefSeq ID NC\_016845.1.

\subsection{Further genome assembly processing.}\label{subsec:proc_ass}
The sample genome and plasmid assemblies were merged, filtered and deduplicated using in-house scripts.
The resulting assemblies were submitted to NCBI.\
The assemblies were annotated locally with Prokka v1.13.7~\cite{Prokka}
% quay.io/biocontainers/prokka:1.13.7--pl526_0
% prokka --compliant --centre UoN --cpu 32 --outdir prokka/sample --force --prefix sample --locustag sample --genus Klebsiella --species pneumoniae sample.fasta
and remotely with the NCBI Prokaryotic Genome Annotation Pipeline (PGAP)~\cite{PGAP}.
The \gls{mlst} results were computed using SRST2 v0.2~\cite{SRST2}
% quay.io/biocontainers/srst2:0.2.0--py27_2
% getmlst.py --species "Klebsiella pneumoniae"
% srst2 --output sample --input_pe sample_cutadapt.1.fastq.gz sample_cutadapt.2.fastq.gz --mlst_db Klebsiella_pneumoniae.fasta --mlst_definitions kpneumoniae.txt --mlst_delimiter '_' --log --threads 32
and Kleborate~\cite{Kleborate}.
The virulence-associated genes encoding yersiniabactin, aerobactin, salmochelin, colibactin, the regulators of mucoid
phenotype, the serotype and the drug resistance determinants were combined using Kleborate with
Kaptive subroutine~\cite{Kaptive}.
% ivasilyev/kleborate_kaptive:latest
% Kleborate --all -o results.tsv -a *.fna
Pangenome analysis was performed across the sequence query containing also 365 \gls{kpne} completed genome assemblies
downloaded from the NCBI FTP server.
A phylogeny was drawn using Roary, the Pan Genome Pipeline v3.12.0~\cite{Roary}.
% sangerpathogens/roary:latest
% roary -p 32 -f roary/ -e --mafft gff/*.gff

\section{Results and discussion}\label{sec:res_dis}
Neonatal infection is a clinical syndrome, characterized by systemic symptoms of first month of life.
However, in this study, we observed the case of newborn infant gastrointestinal tract colonization by the known
pathogen, \gls{kpne}, demonstrating a multi drug resistance phenotype without manifested symptoms.
The microdilution method has confirmed isolates with resistance to aminoglycosides, \betalactam,
nitrofuran, fluoroquinolone, sulfonamide, trimethoprim and fosfomycin antibiotics and \textit{Klebsiella} phage.
All the isolates were susceptible to amikacin, chloramphenicol and pyo bacteriophage.

The discovered in \gls{wgs} \textit{de novo} resistome profile included genetic determinants of drug resistance to
aminoglycoside, fluoroquinolone, macrolide, sulfonamide, chloramphenicol, tetracycline and trimethoprim.
It also confirmed the common \betalactam ase genes spread across the all isolates as well as the occurrence of isolates
carrying certain genes of broad spectrum \betalactam ases with resistance to \betalactam ase inhibitors and \gls{esbl}.
The results of the screening for genetic determinants of resistance to colistin, fosfomycin, glycopeptide,
nitroimidazole, rifampicin and possible production of carbapenemases or \gls{esbl} with resistance to \betalactam ase
inhibitors were negative.

Biotype profiling has revealed 10 isolates and 31 related strain combined into two clusters.
The first cluster included samples \#\# 60, 85, 91, 102 and 24.
Surprisingly, all the isolates except the last one were obtained from the infants born with caesarean delivery.
The other five isolates, \#\# 22, 90, 27, 28, 29, obtained from the children born vaginally,
have formed even less divergent clade.
More interesting, their \gls{mlst} result was only ST23.
All the 5 isolates were also positive for the genes of siderophores yersiniabactin (\textit{ybt}),
aerobactin (\textit{iuc}) and salmochelin (\textit{iro}), genotoxin colibactin (\textit{clb}),
hypermucoidy determinants (\textit{rmpA}, \textit{rmpA2}), specific \textit{wzi} loci alleles related to the
K- (capsule) and O- (LPS) antigens development which cumulative presence corresponds to extremely virulent phenotype
in theory.
In fact, it means the 4 cases of infant intestinal tract colonization by hypervirulent strains of \gls{mdrkp} which
took place silently, without visible symptoms.
The persistence of the \gls{mdrkp} strain has been also confirmed for patient \#1 (sample \#90) at 3rd month of life.

\section{Data availability}\label{sec:data}
The corresponding \gls{wgs} project has been deposited at NCBI GenBank database under the main
BioProject accession ID PRJNA556398.
The project data analysis scripts and materials are hosted on GitHub
(\url{https://github.com/ivasilyev/curated_projects/tree/master/inicolaeva/klebsiella_infants}).




%\pgfplotstabletypeset[
%  string type,
%]{../../datasets/antibiogram.tsv}
% , multicolumn names
%
%\pgfplotstableread[
%  string type, header=true, col sep=tab, multicolumn names,
%]{tsv/phenotype.tsv}\phenotypeWide
%
%
%\pgfplotstabletranspose[string type,
%  % Rename colimns
%  columns/sample_number/.style={column name=Sample \#},
%  columns/delivery/.style={column name=Delivery},
%  columns/patient_id/.style={column name=Patient \#},
%  columns/checkpoint_age_days/.style={column name={Age, d}},
%  columns/checkpoint_kpneumoniae_lg_cfu_per_g/.style={column name={$\lg{CFU} / g$}},
%  columns/extended-spectrum_beta-lactamases/.style={column name=\gls{esbl}},
%  columns/klebsiella_phage/.style={column name=K-phage},
%  % Abbreviation conventions were taken from:
%  % https://aac.asm.org/content/abbreviations-and-conventions
%  columns/amoxicillin-clavulanic acid/.style={column name=AMC},
%  columns/amikacin/.style={column name=AMK},
%  columns/ampicillin/.style={column name=AMP},
%  columns/aztreonam/.style={column name=ATM},
%  columns/ceftriaxone/.style={column name=CRO},
%  columns/ciprofloxacin/.style={column name=CIP},
%  columns/chloramphenicol/.style={column name=CHL},
%  columns/fosfomycin/.style={column name=FOF},
%  columns/gentamicin/.style={column name=GEN},
%  columns/imipenem/.style={column name=IPM},
%  columns/meropenem/.style={column name=MEM},
%  columns/netilmicin/.style={column name=NET},
%  columns/nitrofurantoin/.style={column name=NIT},
%  columns/sulfamethoxazole/.style={column name=SMZ},
%  columns/trimethoprim/.style={column name=TMP},
%]\phenotypeTall{\phenotypeWide}
%
%\pgfplotstabletypeset[string type,
%  every head row/.style={before row=\toprule, after row=\midrule},
%  every last row/.style={after row=\bottomrule},
%  after row=\midrule,
%]{\phenotypeTall}
%
%

%\pgfplotstabletranspose[]\transpose\phenotype
%
%\pgfplotstabletypeset[string type,
%        every head row/.style={before row=\toprule,after row=\midrule},
%        every last row/.style={after row=\bottomrule},
%        after row=\midrule
%]\transpose

% \rotatebox{90}{
%\pgfplotstabletypeset[
%  string type, header=true, col sep=tab, multicolumn names,
%  columns/sample_number/.style={column name=Sample \#},
%  columns/delivery/.style={column name=Delivery},
%  columns/patient_id/.style={column name=Patient \#},
%  columns/checkpoint_age_days/.style={column name={Age, d}},
%  columns/checkpoint_kpneumoniae_lg_cfu_per_g/.style={column name={$\lg{CFU} / g$}},
%  columns/extended-spectrum_beta-lactamases/.style={column name=\gls{esbl}},
%  columns/klebsiella_phage/.style={column name=K-phage},
%  % Abbreviation conventions were taken from:
%  % https://aac.asm.org/content/abbreviations-and-conventions
%  columns/amoxicillin-clavulanic acid/.style={column name=AMC},
%  columns/amikacin/.style={column name=AMK},
%  columns/ampicillin/.style={column name=AMP},
%  columns/aztreonam/.style={column name=ATM},
%  columns/ceftriaxone/.style={column name=CRO},
%  columns/ciprofloxacin/.style={column name=CIP},
%  columns/chloramphenicol/.style={column name=CHL},
%  columns/fosfomycin/.style={column name=FOF},
%  columns/gentamicin/.style={column name=GEN},
%  columns/imipenem/.style={column name=IPM},
%  columns/meropenem/.style={column name=MEM},
%  columns/netilmicin/.style={column name=NET},
%  columns/nitrofurantoin/.style={column name=NIT},
%  columns/sulfamethoxazole/.style={column name=SMZ},
%  columns/trimethoprim/.style={column name=TMP},
%  every head row/.style={before row=\toprule,after row=\midrule},
%  every last row/.style={after row=\bottomrule},
%  after row=\midrule
%]{tsv/phenotype.tsv}
%



\section{Acknowledgments}\label{sec:acks}
TODO

\printbibliography

\newpage
\begin{table}\label{phenotype}

\rotatebox{90}{
\pgfplotstabletypeset[string type,
  string type, header=true, col sep=tab, multicolumn names,
  % Rename colimns
%  columns/sample_number/.style={column name=Sample \#},
%  columns/delivery/.style={column name=Delivery},
%  columns/patient_id/.style={column name=Patient \#},
%  columns/checkpoint_age_days/.style={column name={Age, d}},
%  columns/checkpoint_kpneumoniae_lg_cfu_per_g/.style={column name={$\lg{CFU} / g$}},
%  columns/extended-spectrum_beta-lactamases/.style={column name=\gls{esbl}},
%  columns/klebsiella_phage/.style={column name=K-phage},
%  columns/pyo_bacteriophage/.style={column name=P-phage},
%  % Abbreviation conventions were taken from:
%  % https://aac.asm.org/content/abbreviations-and-conventions
%  columns/amoxicillin-clavulanic acid/.style={column name=AMC},
%  columns/amikacin/.style={column name=AMK},
%  columns/ampicillin/.style={column name=AMP},
%  columns/aztreonam/.style={column name=ATM},
%  columns/ceftriaxone/.style={column name=CRO},
%  columns/ciprofloxacin/.style={column name=CIP},
%  columns/chloramphenicol/.style={column name=CHL},
%  columns/fosfomycin/.style={column name=FOF},
%  columns/gentamicin/.style={column name=GEN},
%  columns/imipenem/.style={column name=IPM},
%  columns/meropenem/.style={column name=MEM},
%  columns/netilmicin/.style={column name=NET},
%  columns/nitrofurantoin/.style={column name=NIT},
%  columns/sulfamethoxazole/.style={column name=SMZ},
%  columns/trimethoprim/.style={column name=TMP},
%  % Rotate columns
  assign column name/.style={/pgfplots/table/column name={\rotatebox{90}{\textbf{#1}}}},
  % Add liners
  every head row/.style={before row=\toprule,after row=\midrule},
  every last row/.style={after row=\bottomrule},
  after row=\midrule,
]{tsv/phenotype.tsv}
}
\end{table}


\newpage
\begin{table}\label{genotype}

\end{table}
\newpage

\vfill
Compiled from \LaTeXe{}.
\end{document}
