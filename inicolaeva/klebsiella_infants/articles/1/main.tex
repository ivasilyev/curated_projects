\documentclass[12pt,a4paper]{article}
\usepackage[utf8]{inputenc}
\usepackage[T1]{fontenc} 
\usepackage[english]{babel}
\usepackage{csquotes}
\usepackage[hidelinks]{hyperref}
\usepackage{authblk}  % Add the affiliation to the author's name
\usepackage{pgfplotstable}  % TSV data support
\usepackage{graphicx}  % Table rotate support
\usepackage{mfirstuc}  % For First Letter Capitalization
\usepackage{booktabs}
\usepackage{makecell}  % Enhance MultiIndex rows
\usepackage[flushleft]{threeparttable}  % For table with footnotes
\usepackage[backend=biber, % style=authoryear,
natbib=true]{biblatex}  % Bibliography support
\usepackage[acronym]{glossaries}  % Abbreviations & species support
\addbibresource{latexrefs.bib}
\nonfrenchspacing

% Define macros
\newcommand{\betalactam}{$\beta$-lactam}
% Rotate & weighten items in table header
\newcommand{\rB}[1]{\rotatebox{90}{\textbf{#1}}}
% Convert cell to the union
\newcommand{\mCL}[1]{\makecell[l]{#1}}

% Define species
% \newacronym{<label>}{<abbrv>}{<full>}
\newacronym[first={\textit{Klebsiella pneumoniae}}]{kpne}{\textit{K. pneumoniae}}{\textit{Klebsiella pneumoniae}}
% Define abbreviations
\newacronym{bla}{BLA}{\betalactam ase}
\newacronym{bsbl}{BSBL}{broad spectrum \glsdesc{bla}}
\newacronym{ibsbl}{BSBL-inhR}{\gls{bsbl} with resistance to \glsdesc{bla} inhibitors}
\newacronym{esbl}{ESBL}{extended spectrum \glsdesc{bla}}
\newacronym{mlst}{MLST}{\textit{in silico} multi-locus sequence type}
\newacronym{mdrkp}{MDRKP}{multi drug resistant \gls{kpne}}
\newacronym{wgs}{WGS}{whole genome shotgun}


% Define authors
\author[1]{Vasilyev, I. Y.} % Corresponding author, e-mail: u0412u0418u042e@gmail.com
\author[2]{Nikolaeva, I. V.}  % irinanicolaeva@mail.ru
\author[1]{Siniagina, M. N.}  % marias25@mail.ru
\author[1]{Kharchenko, A. M.}  % anastasiahm@list.ru
\author[2]{Shaikhieva, G. S.}  % studentgulya@yandex.ru
% Define affilations
\affil[1]{Institute of Fundamental Medicine and Biology, Kazan Federal University, Kazan, Russia}
\affil[2]{Kazan State Medical University, Kazan, Russia}

\title{Multidrug-resistant hypervirulent \Gls{kpne} found persisting silently in infant gut microbiota}
\date{\today}

\begin{document}
\maketitle
\glsresetall

\begin{abstract}
TODO
  It carried virulence genes for yersiniabactin (ybt 1; ICEKp10), colibactin (clb 2), aerobactin (iuc 1), salmochelin (iro 1), and the regulator of mucoid phenotype genes rmpA and rmpA2; it has no resistance genes except for the chromosomal gene blaSHV-11

\end{abstract}

\section{Introduction}\label{sec:intro}
Newborn patients could obtain microorganisms from clinical environment, personnel, other patients and parents,
e.g.\ via breast milk.
Since Enterobacteriaceae are known early human gastrointestinal tract colonisers, local and systemic diseases could take
place during the dynamic microbiome development.
Their persistence as stable colonizers may have destructive consequence on the host vital functions.
Moreover, due to the high horizontal gene transport frequency between the microbiome parts, persistence of even single
strain carrying pathobiotic genes after its spread may cause explicit cyto- and genotoxic effects on host cells
leading to dangerous repercussion including colorectal cancer in particular~\cite{Pope2019}.
Underweight and weakened patients of neonatal intensive care nurseries often suffer Gram-positive bacteria infections,
mostly caused by coagulase-negative Staphylococci, while the infections caused by Gram-negative bacteria are considered
as more rare and deadly after even more rapid generalization~\cite{Dorota2017}.

\gls{kpne} is the causative agent of numerous nosocomial and community acquired infections including
pneumonia, sepsis, bacteremia, meningitis, pyogenic liver abscesses, urinary tract infections and more.
The risk group historically includes patients with weakened and malfunctioning immune system, but the spread of
hypervirulent strains also endangers immunosufficient individuals~\cite{Shankar2018}.
First isolated from lungs of patient with pneumonia \textit{postmortem}, \gls{kpne} were acknowledged as a part of
normal human gastrointestinal tract microbiome since then.
The colonization can spread, persist for years and cause different pathologies from hidden carriage to
fatal acute infections even for the healthy individuals~\cite{Martin2018}.

In this study we applied \gls{wgs} sequencing to describe in detail 8 cases of the asymptomatic carriage of \gls{mdrkp}
by mature infants the gastrointestinal tract detected at similar time periods after the birth without visible affection
on their health.
The source(s) of the \gls{mdrkp} spread was not defined.

\section{Materials and methods}\label{sec:mat_met}
\subsection{Isolation of rectal \gls{mdrkp} samples from the infant patients}\label{subsec:iso}
A total of 10 \gls{kpne} isolates demonstrating multi drug resistance phenotype were collected from 10
stool samples from 8 newborn full-term patients during hospitalization in maternity hospital of Kazan, Russia.
5 infants were born with vaginal delivery, 3~--- after cesarean surgery.
No infection outbreak was recorded and no detailed microbiological analysis was performed over parents.
The infant stool samples were collected at 3~--- 4 day of life and contaminated with $10^8$~--- $10^9$ of \gls{mdrkp}
colony-forming units per gram of stool.
The meconium samples obtained from the all 8 individuals were not tested for contamination.
Yet, the meconium samples obtained from 40 infants during the previous research were sterile or
contained lactobacteria~\cite{Nikolaeva2019a}.
Two samples collected from two infants after 1 and 6 months of monitoring also contained \gls{mdrkp}.
All the 8 babies were discharged from hospital at 4~--- 5 day of life in satisfactory condition.
Blood in stool and liquid stool were reported once for patient \#1 to the third month of life,
and constipation was reported for patient \#2 to the first month of life.
No other manifested and prolonged symptoms were reported across the whole monitoring time from the all individuals.

\subsection{Phenotypic Characterization}\label{subsec:phe}
The antimicrobial susceptibilities of the 10 microbial isolates were determined using a broth microdilution procedure.
The following antibacterial agents were tested: aminoglycosides (amikacin, netilmicin, gentamicin),
\betalactam s (amoxicillin-clavulanic acid, ampicillin, aztreonam, ceftriaxone, imipenem, meropenem),
nitrofuran derivatives (nitrofurantoin), sulfonamides (sulfamethoxazole), 2,4-diaminopyrimidines (trimethoprim),
fluoroquinolones (ciprofloxacin), chloramphenicol, fosfomycin.
The production of \gls{esbl} and susceptibility to \textit{Klebsiella} phage and pyo bacteriophage
were also analyzed during the routine.
The results were interpreted in automated mode using VITEK 2 Compact analyzer (bioMérieux SA, France) according to
producer's guidance documents.

\subsection{Whole-genome sequencing and assembly.}\label{subsec:proc_raw}
Libraries were prepared using Nextera XT DNA Library Preparation Kit.
Whole-genome DNA was sequenced using Illumina MiSeq platform (Illumina Inc., USA),
with a paired-end run of 2 by 250 bp.
Raw reads quality control was performed with FastQC v0.11~\cite{FastQC},  % quay.io/biocontainers/fastqc:0.11.8--1
% fastqc -t 32 sample.fastq.gz -o sample
trimmed by Trimmomatic v0.39~\cite{Trimmomatic}
% quay.io/biocontainers/trimmomatic:0.39--1
% trimmomatic PE -threads 32 -phred33 sample.1.fastq.gz sample.2.fastq.gz sample_trimmomatic.1.fastq.gz sample_trimmomatic_untrimmed.1.fastq.gz sample_trimmomatic.2.fastq.gz sample_trimmomatic_untrimmed.2.fastq.gz ILLUMINACLIP:adapters.fasta:2:30:10 LEADING:3 TRAILING:3 SLIDINGWINDOW:4:15 MINLEN:36
and Cutadapt v2.4~\cite{Cutadapt},
% quay.io/biocontainers/cutadapt:2.4--py37h14c3975_0
% cutadapt -a AGATCGGAAGAG -A AGATCGGAAGAG -m 50 -o sample_cutadapt.1.fastq.gz -p sample_cutadapt.2.fastq.gz sample_trimmomatic.1.fastq.gz sample_trimmomatic.2.fastq.gz
assembled using SPAdes v3.9.1~\cite{SPAdes}.
% quay.io/biocontainers/spades:3.9.1--0
% spades --careful -o sample/genome -1 sample_cutadapt.1.fastq.gz -2 sample_cutadapt.2.fastq.gz
% spades --careful -o sample/plasmid -1 sample_cutadapt.1.fastq.gz -2 sample_cutadapt.2.fastq.gz --plasmid
Assembly statistics were calculated using the reference \gls{kpne} genome with RefSeq ID NC\_016845.1.

\subsection{Further genome assembly processing.}\label{subsec:proc_ass}
The per-sample chromosome and plasmid assemblies were merged, filtered and deduplicated using in-house scripts.
The resulting assemblies were submitted to NCBI.\
The assemblies were annotated locally with Prokka v1.13.7~\cite{Prokka}
% quay.io/biocontainers/prokka:1.13.7--pl526_0
% prokka --compliant --centre UoN --cpu 32 --outdir prokka/sample --force --prefix sample --locustag sample --genus Klebsiella --species pneumoniae sample.fasta
and remotely with the NCBI Prokaryotic Genome Annotation Pipeline (PGAP)~\cite{PGAP}.
The \gls{mlst} results were computed using SRST2 v0.2~\cite{SRST2}
% quay.io/biocontainers/srst2:0.2.0--py27_2
% getmlst.py --species "Klebsiella pneumoniae"
% srst2 --output sample --input_pe sample_cutadapt.1.fastq.gz sample_cutadapt.2.fastq.gz --mlst_db Klebsiella_pneumoniae.fasta --mlst_definitions kpneumoniae.txt --mlst_delimiter '_' --log --threads 32
and Kleborate~\cite{Kleborate}.
The virulence-associated genes encoding yersiniabactin, aerobactin, salmochelin, colibactin, the regulators of mucoid
phenotype, the serotype and the drug resistance determinants were combined using Kleborate with
Kaptive subroutine~\cite{Kaptive}.
% ivasilyev/kleborate_kaptive:latest
% Kleborate --all -o results.tsv -a *.fna
Pangenome analysis was performed across the sequence query containing also 365 \gls{kpne} completed genome assemblies
downloaded from the NCBI FTP server.
A phylogeny was drawn using Roary, the Pan Genome Pipeline v3.12.0~\cite{Roary}.
% sangerpathogens/roary:latest
% roary -p 32 -f roary/ -e --mafft gff/*.gff

\section{Results and discussion}\label{sec:res_dis}
Neonatal infection is a clinical syndrome, characterized by systemic symptoms of first month of life.
However, in this study we observed the case of newborn infant gastrointestinal tract colonization by the known
pathogen, \gls{kpne}, demonstrating a multi drug resistance phenotype without manifested symptoms.

The microdilution method has confirmed isolates with resistance to aminoglycosides, \betalactam,
nitrofuran, fluoroquinolones, sulfonamides, trimethoprim and fosfomycin antibiotics and \textit{Klebsiella} phage.
All the isolates were susceptible to amikacin, chloramphenicol and pyo bacteriophage~\ref{tab:phenotype}.

The discovered in \gls{wgs} \textit{de novo} resistome profile included genetic determinants of drug resistance to
aminoglycosides, fluoroquinolones, macrolides, sulfonamides, chloramphenicol, tetracycline and trimethoprim.
It also confirmed the common \gls{bla} genes spread across the all isolates as well as the occurrence of isolates
carrying certain genes of \gls{bsbl}, \gls{ibsbl} and \gls{esbl}~\ref{tab:genotype}.
The results of the screening for genetic determinants of resistance to colistin, fosfomycin, glycopeptide,
nitroimidazole, rifampicin and possible production of carbapenemases or \gls{esbl} with resistance to \glsdesc{bla}
inhibitors were negative.

Biotype profiling has revealed that 10 isolates were related to 31 reference strain and combined into two clusters.
The first cluster has included samples \#\# 60, 85, 91, 102 and 24.
Surprisingly, all the isolates except the last one were obtained from the infants born with cesarean delivery.
The \gls{mlst} analysis results have included STs 37, 45, 268 and 983\-1LV.\
Due to the genetic divergence within the group, the patients most likely were infected from different sources.
Since the lack of the microbiological data from detailed examination performed over their mothers, we also cannot except
the vertical transmission case, e.g.\ after the silent colonization of maternal urinary tract by \gls{mdrkp}.
It is known that asymptomatic bacteriuria occurs in 2~--- 7 \% of pregnant women in the United States
and may cause severe urinary tract infection and pyelonephritis even leading to nephrectomy~\cite{Kim2018}.
Gravidas are experiencing 20-fold increased risk of pyelonephritis mainly caused by the pregnancy immunosuppression,
mechanical bladder compression and ureteral dilatation~\cite{Farkash2012}.
A vertical transmission of pathogenic microorganisms is possible during the birth and before.
The described cases of uterine \gls{kpne} infection during pregnancy include penetration via fetal membranes and
hemato-placental barrier, resulting to chorioamnionitis~\cite{Oh2017} and acute placental infection~\cite{Sheikh2005}.
The infection source for the first group of infants remains unclear, because of the fact that all the women
were reported as healthy, with no symptoms or complains.

The other five isolates, \#\# 22, 90, 27, 28, 29, obtained from the infants born vaginally,
have formed even less divergent clade, allowing us to make a conclusion about a single \gls{mdrkp} infection source.
All the 5 isolates were also positive for the genes of siderophores yersiniabactin (\textit{ybt}),
aerobactin (\textit{iuc}) and salmochelin (\textit{iro}), genotoxin colibactin (\textit{clb}),
hypermucoidy determinants (\textit{rmpA}, \textit{rmpA2}), specific \textit{wzi} loci alleles related to the
K- (capsule) and O- (LPS) antigens development which cumulative presence corresponds to extremely virulent phenotype
in theory.
More important, their \gls{mlst} was only ST23.
First reported in the mid-1980s in Taiwan, ST23 was often mentioned to present and strongly associated with the
K1 capsular serotype~\cite{Shon2013}.
The \gls{kpne} ST23 strains were confirmed in numeric researches as multidrug-resistant hypervirulent pathogens
caused abscesses of kidneys, pancreas, liver~\cite{Shen2019}, endogenous endophthalmitis~\cite{Xu2018},
with a dominance in the Asian-Pacific region~\cite{Thiry2019}.
In fact, it means the 4 cases of asymptomatic colonization of infant intestinal tract by hypervirulent strains of
\gls{mdrkp} which were spread a short time ago and may be accounted as a hospital-acquired infection.
The persistence of the \gls{mdrkp} ST23 strain has been also confirmed for the patient \#1 (sample \#90)
at 3rd month of life~\ref{tab:phenotype}~--- still without acute immune response.

The fetal immune system is not developed compared with later life mostly due to the environmental limitation~---
the stress of the own remodelling tissues and the non-inherited maternal alloantigens should not provoke
the immune reaction from the fetus side~\cite{Simon2015}.
Their innate immune system is muted~\cite{Haase2010}, the humoral immune responses are blunted,
the immunoglobulin class switching is incomplete, the released IgG antibodies decline rapidly after
immunization~\cite{Pihlgren2006}.
However, the immature Th1-type T-cell response is compensated by the IFN-$\gamma$ producing
$\gamma\delta$-T-cells~\cite{Gibbons2009}.
By the way, the main mechanism of neonatal immune protection is based on vertical transport of antibodies
via breastfeeding.
With some fortune, breastfed newborns will not suffer infections that have induced the immune response
from their mothers earlier.
Otherwise, a lack of specific antibodies will result an infection development~\cite{Dias2017}.

The demonstrated phenomenon of silent carriage of classic and hypervirulent \gls{kpne} strains within an
infant gut microbiome may indicate the successful production of the required antibodies by a maternal organism.
It is possible after immunisation occurred definitely before the childbirth~--- in a maternity hospital or even earlier.

\section{Data availability}\label{sec:data}
The corresponding \gls{wgs} project has been deposited at NCBI GenBank database under the main
BioProject accession ID PRJNA556398.
The project data analysis scripts and materials are hosted on GitHub
(\url{https://github.com/ivasilyev/curated_projects/tree/master/inicolaeva/klebsiella_infants}).

\section{Acknowledgments}\label{sec:acks}
TODO

\printbibliography

\newpage
\begin{table}
\begin{threeparttable}

\caption{Patient data, sample data and demonstrated drug resistance of the studying \gls{kpne} isolates}
\label{tab:phenotype}
\centering
\noindent
\begin{tabularx}{\textwidth}{lllllllllll}
\toprule
                       Sample name & \rB{Kleb102} & \rB{Kleb22} & \rB{Kleb24} & \rB{Kleb27} & \rB{Kleb28} & \rB{Kleb29} & \rB{Kleb60} & \rB{Kleb85} & \rB{Kleb90} & \rB{Kleb91} \\
\midrule
                      Sample number &         102 &          22 &          24 &          27 &     28 &     29 &     60 &     85 &     90 &     91 \\
                           Delivery &           C &           V &           V &           V &      V &      V &      C &      C &      V &      C \\
                         Patient ID &           2 &           1 &           3 &           4 &      5 &      6 &      7 &      8 &      1 &      2 \\
                  Patient age, days &          33 &           3 &           3 &           3 &      4 &      3 &      4 &      4 &     90 &      3 \\
  \mCL{\gls{kpne},\\$\lg{CFU / g}$} &           7 &           9 &           8 &           8 &      9 &      8 &      9 &      9 &      8 &      8 \\
                         \gls{esbl} &           - &           + &           - &           - &      + &      + &      + &      + &      - &      + \\
\midrule
          \textit{Klebsiella} phage &           R &           S &           R &           R &      N &      R &      S &      S &      R &      R \\
                          Pyo phage &           S &           S &           S &           S &      S &      S &      S &      S &      S &      S \\
                           Amikacin &           S &           S &           S &           S &      S &      S &      S &      S &      S &      S \\
\mCL{Amoxicillin-\\clavulanic acid} &           R &           R &           S &           S &      R &      R &      R &      R &      S &      R \\
                         Ampicillin &           R &           R &           R &           R &      R &      R &      R &      R &      R &      R \\
                          Aztreonam &           R &           R &           S &           S &      R &      R &      R &      R &      N &      R \\
                        Ceftriaxone &           R &           R &           S &           S &      R &      R &      R &      R &      S &      R \\
                    Chloramphenicol &           N &           S &           S &           S &      S &      S &      S &      S &      S &      S \\
                      Ciprofloxacin &           S &           I &           S &           S &      R &      I &      S &      S &      S &      S \\
                         Fosfomycin &           R &           R &           I &           R &      R &      R &      R &      R &      N &      R \\
                         Gentamicin &           S &           S &           S &           N &      R &      R &      S &      S &      N &      S \\
                           Imipenem &           R &           S &           S &           I &      R &      I &      R &      R &      N &      R \\
                          Meropenem &           R &           S &           S &           S &      S &      S &      S &      S &      N &      S \\
                         Netilmicin &           S &           R &           S &           S &      R &      R &      S &      S &      N &      I \\
                     Nitrofurantoin &           R &           I &           S &           R &      R &      I &      R &      R &      R &      R \\
                   Sulfamethoxazole &           S &           R &           S &           S &      R &      R &      R &      R &      N &      R \\
                       Trimethoprim &           S &           R &           S &           S &      R &      R &      R &      R &      N &      R \\
\bottomrule
\end{tabularx}

\begin{tablenotes}
\item
Delivery: V~--- vaginal, C~--- cesarean section.
Susceptibility to the antimicrobial agent: S~-- susceptible, R~-- resistant, I~-- intermediate, N~-- not defined.
\end{tablenotes}
\end{threeparttable}
\end{table}


\newpage
\begin{sidewaystable}[ht]
\begin{threeparttable}
\tiny

\caption{Genome assembly data, \Acrshort{mlst}s and drug resistance determinants of the \gls{kpne} isolates}
\label{tab:genotype}
\centering
\noindent
\setlength\tabcolsep{1.5pt}  % Default value: 6pt l@{\hspace{1.5\tabcolsep}
\begin{tabularx}{\textwidth}{lllllllllll}
\toprule
                   Sample name &       \textbf{Kleb102} &                  \textbf{Kleb22} &   \textbf{Kleb24} &          \textbf{Kleb27} &                 \textbf{Kleb28} &                \textbf{Kleb29} &                  \textbf{Kleb60} &         \textbf{Kleb85} &                  \textbf{Kleb90} &         \textbf{Kleb91} \\
\midrule
                  Accession ID &           VOOJ00000000 &                     VOOA00000000 &      VOOC00000000 &             VOOD00000000 &                    VOOE00000000 &                   VOOF00000000 &                     VOOG00000000 &            VOOH00000000 &                     VOOB00000000 &            VOOI00000000 \\
                  Reads number &                1040088 &                           805318 &            856300 &                  1056762 &                          926382 &                        1259976 &                           994296 &                 1207354 &                          1216382 &                 1381726 \\
                      Coverage &                  58.7x &                            45.4x &             48.3x &                    59.6x &                           52.3x &                          71.1x &                            56.1x &                   68.1x &                            68.6x &                   78.0x \\
\mCL{Genome\\assembly\\length} &                5297684 &                          5881113 &           5518356 &                  5641111 &                         5885411 &                        5890967 &                          5666743 &                 5324705 &                          5882697 &                 5326604 \\
                Contigs number &                    106 &                              111 &                94 &                       76 &                             120 &                            118 &                               91 &                      74 &                              108 &                      77 \\
                    N50 metric &                 350001 &                           307112 &            319410 &                   322041 &                          307112 &                         307112 &                           263911 &                  358425 &                           307112 &                  437151 \\
 \mCL{Largest\\contig\\length} &                1106296 &                           770102 &            930837 &                  1334922 &                          771547 &                         799268 &                           642551 &                  812552 &                           799226 &                  812552 \\
\midrule
                    \gls{mlst} &              ST983-1LV &                             ST23 &              ST37 &                     ST23 &                            ST23 &                           ST23 &                            ST268 &                    ST45 &                             ST23 &                    ST45 \\
                Yersiniabactin &                   ybt? &            \mCL{ybt 1,\\ICEKp10} &                 - &    \mCL{ybt 1,\\ICEKp10} &           \mCL{ybt 1,\\ICEKp10} &          \mCL{ybt 1,\\ICEKp10} &           \mCL{ybt 17,\\ICEKp10} &                       - &            \mCL{ybt 1,\\ICEKp10} &                       - \\
                    Colibactin &                      - &                            clb 2 &                 - &                    clb 2 &                           clb 2 &                          clb 2 &                            clb 3 &                       - &                            clb 2 &                       - \\
                    Aerobactin &                      - &                            iuc 1 &                 - &                    iuc 1 &                           iuc 1 &                          iuc 1 &                            iuc 1 &                       - &                            iuc 1 &                       - \\
                   Salmochelin &                      - &                            iro 1 &                 - &                    iro 1 &                           iro 1 &                          iro 1 &                            iro 1 &                       - &                            iro 1 &                       - \\
                 \textit{rmpA} &                      - &         \mCL{rmpA$_2$\\(KpVP-1)} &                 - & \mCL{rmpA$_2$\\(KpVP-1)} &        \mCL{rmpA$_2$\\(KpVP-1)} &       \mCL{rmpA$_2$\\(KpVP-1)} &        \mCL{rmpA$_2$*\\(KpVP-1)} &                       - &         \mCL{rmpA$_2$\\(KpVP-1)} &                       - \\
                \textit{rmpA2} &                      - &                        rmpA2$_8$ &                 - &                rmpA2$_5$ &                       rmpA2$_8$ &                      rmpA2$_8$ &                       rmpA2$_2$* &                       - &                        rmpA2$_8$ &                       - \\
                  \textit{wzi} &                  wzi39 &                             wzi1 &            wzi123 &                     wzi1 &                            wzi1 &                           wzi1 &                            wzi95 &                  wzi101 &                             wzi1 &                  wzi101 \\
                       K locus &                   KL39 &                              KL1 &             KL136 &                      KL1 &                             KL1 &                            KL1 &                             KL20 &                    KL24 &                              KL1 &                    KL24 \\
                       O locus &                    O3b &                             O1v2 &              O2v2 &                     O1v2 &                            O1v2 &                           O1v2 &                             O2v1 &                    O2v1 &                             O1v2 &                    O2v1 \\
               Aminoglycosides &     \mCL{StrB,\\StrA*} &   \mCL{StrB,\\StrA*,\\Aac3-IIa*} & \mCL{StrB,\\StrA} &                        - &  \mCL{StrB,\\StrA*,\\Aac3-IIa*} & \mCL{StrB,\\StrA*,\\Aac3-IIa*} &    \mCL{StrB,\\StrA*,\\Aph3-Ia*} &      \mCL{StrB,\\StrA*} &   \mCL{StrB,\\StrA*,\\Aac3-IIa*} &      \mCL{StrB,\\StrA*} \\
              Fluoroquinolones &                      - &                           QnrB1? &                 - &                        - &                          QnrB1? &                         QnrB1? &                                - &                       - &                           QnrB1? &                       - \\
                    Macrolides &                      - &                                - &                 - &                        - &                               - &                              - &                            EreA2 &                       - &                                - &                       - \\
                     Phenicols &                 CatA1* &                            CatB4 &                 - &                        - &                           CatB4 &                          CatB4 &                                - &                   CatB4 &                            CatB4 &                   CatB4 \\
                  Sulfonamides &                 SulII* &              \mCL{SulII,\\SulII} &                 - &                        - &                           SulII &                          SulII &              \mCL{SulI,\\SulII*} &                   SulII &                            SulII &                   SulII \\
                 Tetracyclines &                   TetA &                             TetA &                 - &                        - &                            TetA &                           TetA &                                - &                       - &                             TetA &                       - \\
                  Trimethoprim &                      - &                           DfrA14 &                 - &                        - &                          DfrA14 &                         DfrA14 &                            DfrA5 &                  DfrA14 &                           DfrA14 &                  DfrA14 \\
                     \gls{bla} & \mCL{SHV-187*,\\AmpH*} &    \mCL{SHV-190*,\\AmpH,\\OXA-1} &             AmpH* &    \mCL{SHV-190*,\\AmpH} &   \mCL{SHV-190*,\\AmpH,\\OXA-1} &  \mCL{SHV-190*,\\AmpH,\\OXA-1} &                            AmpH* &     \mCL{AmpH*,\\OXA-1} &    \mCL{SHV-190*,\\AmpH,\\OXA-1} &     \mCL{AmpH*,\\OXA-1} \\
                    \gls{esbl} &                      - &                         CTX-M-15 &                 - &                        - &                        CTX-M-15 &                       CTX-M-15 &           \mCL{SHV-13*, CTX-M-3} &                CTX-M-15 &                         CTX-M-15 &                CTX-M-15 \\
                    \gls{bsbl} &                      - &                                - &           SHV-77* &                        - &                               - &                              - &                                - &                       - &                                - &                       - \\
                   \gls{ibsbl} &                TEM-30* &                          TEM-30* &                 - &                        - &                         TEM-30* &                        TEM-30* &                                - & \mCL{SHV-26*,\\TEM-30*} &          \mCL{TEM-30*,\\TEM-30*} & \mCL{SHV-26*,\\TEM-30*} \\
\bottomrule
\end{tabularx}

\begin{tablenotes}
\item
\gls{mlst}~--- \glsdesc{mlst}, \gls{bla}~--- \glsdesc{bla}, \gls{bsbl}~--- \glsdesc{bsbl},
\gls{ibsbl}~--- \glsdesc{ibsbl}, \gls{esbl}~--- \glsdesc{esbl}.
\end{tablenotes}
\end{threeparttable}
\end{sidewaystable}


\newpage

\vfill
Compiled from \LaTeXe{}.
\end{document}
