\begin{abstract}
\begin{minipage}{\linewidth}

\paragraph{Introduction}
% \gls{kpne} is a common microorganism and a causative agent for nosocomial infections.
Since the spread of \gls{mdrkp} strains is considered as a challenge for patients with weakened immunity,
the emergence of isolates carrying determinants of hyper virulent phenotype in addition may become a serious
problem even for healthy individuals.

\paragraph{Materials and methods.}
10 isolates were collected from 8 term neonates in maternity hospital of Kazan, Russia.
All the infants and their mothers were dismissed without complains, in satisfactory condition.
The drug resistance has been defined using a microdilution technique.
\Gls{wgs} sequencing has been performed to obtain detailed information.

\paragraph{Results.}
Phenotype analysis has confirmed production of \gls{esbl} and resistance to aminoglycosides, \betalactam,
nitrofurantoin, fluoroquinolones, sulfonamides, trimethoprim and fosfomycin antibiotics and \textit{Klebsiella} phage.
The \gls{wgs} analysis has revealed genes of resistance to aminoglycosides, fluoroquinolones, macrolides,
sulfonamides, chloramphenicol, tetracycline and trimethoprim and \gls{esbl} determinants.
The pangenome analysis had split the isolates into two phylogenetic clades.
The first, more heterogeneous clade, was represented by 5 isolates with 4 different \gls{mlst}s.
The second group contained 5 isolates from infants born vaginally with the single \gls{mlst}, ST23, positive for
genes corresponding to hyper virulent phenotype: yersiniabactin, aerobactin, salmochelin, colibactin, hypermucoidy
determinants, specific alleles of K- and O-antigens.

\paragraph{Conclusion.}
The spread of hyper virulent \gls{mdrkp} strains is a severe threat especially in a maternity hospital settings.
Therefore, this study demonstrates the case of infected infant organisms being capable to lessen the severity of disease
to asymptomatic carriage.
Since \textit{Klebsiella spp.} are known inducers of specific humoral immunity, the required antibodies
seem to be obtained by newborns vertically, from their mothers, which were immunised a time ago.

\paragraph{Keywords.}
Neonate immunity, \glsfirst{kpne}, \glsdesc{wgs}, \glsdesc{mlst}
\glsresetall

\end{minipage}
\end{abstract}
