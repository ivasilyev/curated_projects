\section{Introduction}\label{sec:intro}
Newborn patients could obtain microorganisms from clinical environment, personnel, other patients and parents,
e.g.\ via breast milk.
Since Enterobacteriaceae are known early human gastrointestinal tract colonisers, local and systemic diseases could take
place during the dynamic microbiome development.
Their stabilization and persistence as may have destructive consequence on the host vital functions.
Moreover, due to the high horizontal gene transport frequency between the microbiome members, persistence of even single
strain carrying pathobiotic genes after its spread may cause explicit cyto- and genotoxic effects on host cells
leading to dangerous repercussion including colorectal cancer in particular~\cite{Pope2019}.
Underweight and weakened patients of neonatal intensive care nurseries often suffer Gram-positive bacteria infections,
mostly caused by coagulase-negative Staphylococci, while the infections caused by Gram-negative bacteria are considered
as more rare and deadly after even more rapid generalization~\cite{Dorota2017}.

\gls{kpne} is the causative agent of numerous nosocomial and community acquired infections including
pneumonia, sepsis, bacteremia, meningitis, pyogenic liver abscesses, urinary tract infections and more.
The risk group historically includes patients with weakened and malfunctioning immune system, but the spread of
hypervirulent strains also endangers immunosufficient individuals~\cite{Shankar2018}.
First isolated from lungs of patient with pneumonia \textit{postmortem}, \gls{kpne} were acknowledged as a part of
normal human gastrointestinal tract microbiome since then.
The colonization can spread, persist for years and cause different pathologies from hidden carriage to
fatal acute infections even for the healthy individuals~\cite{Martin2018}.

In this study we applied \gls{wgs} sequencing to describe in detail 8 cases of asymptomatic carriage of \gls{mdrkp}
in the gastrointestinal tract of term infants, detected at similar time periods after the birth
without visible affection on their health.
