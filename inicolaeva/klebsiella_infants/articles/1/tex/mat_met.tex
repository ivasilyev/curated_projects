\section{Materials and methods}\label{sec:mat_met}
\subsection{Ethical approval}\label{subsec:eth}
The infant stool samples collection was performed as a routine task during normal observation at the
maternity hospital.
The standard biosecurity and institutional safety procedures have been adhered to handle human
subjects.
Informed consent was obtained from all individual participants’ parents included in the study.
No special ethical restrictions were raised.

\subsection{Isolation of stool \gls{mdrkp} samples from the infant patients}\label{subsec:iso}
A total of 10 \gls{kpne} isolates demonstrating multi drug resistance phenotype were collected from
10 stool samples (8 non-repetitive) from 8 newborn full-term breastfed infants during hospitalization in
maternity hospital of Kazan, Russia.
5 neonates were born with vaginal delivery, 3~--- after cesarean surgery.
No infection outbreak was recorded and no detailed microbiological analysis was performed over parents.
The infant stool samples were collected at 3~--- 4 day of life and contaminated with $10^8$~--- $10^9$ of \gls{mdrkp}
colony-forming units per gram of stool.
The meconium samples obtained from the all 8 individuals were not tested for contamination.
Yet, the meconium samples obtained from 40 infants during the previous research were conditionally sterile or
contained bifidobacteria~\cite{Nikolaeva2019a}.
Two samples collected from two infants after 1 and 3 months of monitoring also contained \gls{mdrkp}~\refTab{phenotype}.
All the 8 neonates were discharged from hospital at 4~--- 5 day of life in satisfactory condition.
Blood in stool and liquid stool were reported only once for the patient \#1 to the third month of life,
and constipation was reported for the patient \#2 to the first month of life.
No other manifested and prolonged symptoms were reported across the whole monitoring time from the all individuals.

\subsection{Phenotype Characterization}\label{subsec:phe}
The antimicrobial susceptibilities of the 10 microbial isolates were determined using a broth microdilution procedure.
The following antibacterial agents were tested: aminoglycosides (amikacin, netilmicin, gentamicin),
\betalactam s (amoxicillin-clavulanic acid, ampicillin, aztreonam, ceftriaxone, imipenem, meropenem),
nitrofuran derivatives (nitrofurantoin), sulfonamides (sulfamethoxazole), 2,4-diaminopyrimidines (trimethoprim),
fluoroquinolones (ciprofloxacin), chloramphenicol, fosfomycin.
The production of \gls{esbl} and susceptibility to \textit{Klebsiella} phage and pyo bacteriophage
were also analyzed during the routine.
The results were interpreted in automated mode using VITEK 2 Compact analyzer (bioMérieux SA, France) according to
producer's guidance documents.

\subsection{Whole-genome sequencing and assembly}\label{subsec:proc_raw}
Libraries were prepared using Nextera XT DNA Library Preparation Kit.
Whole-genome DNA was sequenced using Illumina MiSeq platform (Illumina Inc., USA),
with a paired-end run of 2 by 250 bp.
Raw reads quality control was performed with FastQC v0.11~\cite{FastQC},  % quay.io/biocontainers/fastqc:0.11.8--1
% fastqc -t 32 sample.fastq.gz -o sample
then reads were trimmed using Trimmomatic v0.39~\cite{Trimmomatic}
% quay.io/biocontainers/trimmomatic:0.39--1
% trimmomatic PE -threads 32 -phred33 sample.1.fastq.gz sample.2.fastq.gz sample_trimmomatic.1.fastq.gz sample_trimmomatic_untrimmed.1.fastq.gz sample_trimmomatic.2.fastq.gz sample_trimmomatic_untrimmed.2.fastq.gz ILLUMINACLIP:adapters.fasta:2:30:10 LEADING:3 TRAILING:3 SLIDINGWINDOW:4:15 MINLEN:36
and Cutadapt v2.4~\cite{Cutadapt},
% quay.io/biocontainers/cutadapt:2.4--py37h14c3975_0
% cutadapt -a AGATCGGAAGAG -A AGATCGGAAGAG -m 50 -o sample_cutadapt.1.fastq.gz -p sample_cutadapt.2.fastq.gz sample_trimmomatic.1.fastq.gz sample_trimmomatic.2.fastq.gz
assembled using SPAdes v3.9.1~\cite{SPAdes}.
% quay.io/biocontainers/spades:3.9.1--0
% spades --careful -o sample/genome -1 sample_cutadapt.1.fastq.gz -2 sample_cutadapt.2.fastq.gz
% spades --careful -o sample/plasmid -1 sample_cutadapt.1.fastq.gz -2 sample_cutadapt.2.fastq.gz --plasmid
Assembly statistics were calculated using the reference \gls{kpne} genome with RefSeq ID NC\_016845.1.

\subsection{Further genome assembly processing}\label{subsec:proc_ass}
The per-sample chromosome and plasmid assemblies were merged, filtered and deduplicated using in-house scripts.
The resulting assemblies were submitted to NCBI.\
The assemblies were annotated locally with Prokka v1.13.7~\cite{Prokka}
% quay.io/biocontainers/prokka:1.13.7--pl526_0
% prokka --compliant --centre UoN --cpu 32 --outdir prokka/sample --force --prefix sample --locustag sample --genus Klebsiella --species pneumoniae sample.fasta
and remotely with the NCBI Prokaryotic Genome Annotation Pipeline (PGAP)~\cite{PGAP}.
The \gls{mlst} results were computed using SRST2 v0.2~\cite{SRST2}
% quay.io/biocontainers/srst2:0.2.0--py27_2
% getmlst.py --species "Klebsiella pneumoniae"
% srst2 --output sample --input_pe sample_cutadapt.1.fastq.gz sample_cutadapt.2.fastq.gz --mlst_db Klebsiella_pneumoniae.fasta --mlst_definitions kpneumoniae.txt --mlst_delimiter '_' --log --threads 32
and Kleborate~\cite{Kleborate}.
The virulence-associated genes encoding yersiniabactin, aerobactin, salmochelin, colibactin, the regulators of mucoid
phenotype, the serotype and the drug resistance determinants were combined using Kleborate with
Kaptive subroutine~\cite{Kaptive}.
% ivasilyev/kleborate_kaptive:latest
% Kleborate --all -o results.tsv -a *.fna
Pangenome analysis was performed across the sequence query containing also 365 \gls{kpne} completed genome assemblies
downloaded from the NCBI FTP server.
A phylogeny was drawn using Roary, the Pan Genome Pipeline v3.12.0~\cite{Roary}.
% sangerpathogens/roary:latest
% roary -p 32 -f roary/ -e --mafft gff/*.gff
