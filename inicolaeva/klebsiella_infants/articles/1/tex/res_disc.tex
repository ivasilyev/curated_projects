\section{Results and discussion}\label{sec:res_dis}
Neonatal infection is a clinical syndrome, characterized by systemic symptoms of first month of life.
However, in this study we observed the case of newborn infant gastrointestinal tract colonization by the known
pathogen, \gls{kpne}, demonstrating a multi drug resistance phenotype without manifested symptoms.

The microdilution method has confirmed isolates with resistance to aminoglycosides, \betalactam,
nitrofuran, fluoroquinolones, sulfonamides, trimethoprim and fosfomycin antibiotics and \textit{Klebsiella} phage.
All the isolates were susceptible to amikacin, chloramphenicol and pyo bacteriophage~\refTab{phenotype}.

The discovered in \gls{wgs} \textit{de novo} resistome profile has included genetic determinants of drug resistance to
aminoglycosides, fluoroquinolones, macrolides, sulfonamides, chloramphenicol, tetracycline and trimethoprim.
It also confirmed the common \gls{bla} genes spread across the all isolates as well as the occurrence of isolates
carrying certain genes of \gls{bsbl}, \gls{ibsbl} and \gls{esbl}~\refTab{genotype}.
The results of the screening for genetic determinants of resistance to colistin, fosfomycin, glycopeptide,
nitroimidazole, rifampicin and possible production of carbapenemases or \gls{esbl} with resistance to \gls{bla}
inhibitors were negative.

Biotype profiling has revealed that 10 isolates were related to 31 reference strain and combined into two
clusters~\refFig{tree}.
The first cluster has included samples \#\# 60, 85, 91, 102 and 24.
All the isolates except the last one were obtained from the infants born with cesarean delivery.
The \gls{mlst} analysis results have included the STs 37, 45, 268 and 983\-1LV.~\refTab{genotype}
The ST37 strains were reported as possible reservoirs for carbapenem resistance genes during antimicrobial therapy
in neonates~\cite{Li2017}, the ST45~--- as the \gls{esbl}-carrying infectious agent of highly-contagious
neonatal sepsis~\cite{Marando2018}, the ST268~--- as possible reservoirs for New Delhi metallo-\gls{bla}
(NDM)~\cite{Kiaei2019}, the ST983~--- as typical causative agents of nosocomial infections
also harboring \gls{esbl}~\cite{Founou2019}.

Due to the genetic divergence within the group, the patients most likely were infected from different sources.
Since the lack of the microbiological data from detailed examination performed over their mothers, we also cannot except
the vertical transmission case, e.g.\ after the silent colonization of maternal urinary tract by \gls{mdrkp}.
It is known that asymptomatic bacteriuria occurs in 2~--- 7 \% of pregnant women in the United States
and may cause severe urinary tract infection even leading to nephrectomy~\cite{Kim2018}.
Gravidas are experiencing 20-fold increased risk of pyelonephritis mainly caused by pregnancy immunosuppression,
mechanical bladder compression and ureteral dilatation~\cite{Farkash2012}.
A vertical transmission of pathogenic microorganisms is possible during the birth and before.
The described cases of uterine \gls{kpne} infection during pregnancy include penetration via fetal membranes and
hemato-placental barrier, resulting to chorioamnionitis~\cite{Oh2017} and acute placental infection~\cite{Sheikh2005}.
The infection source for the first group of studying infants remains unclear, due to the fact that their mothers
were reported as healthy, with no symptoms or complains.

The other five isolates, \#\# 22, 90, 27, 28, 29, obtained from the infants born vaginally, have formed even less
divergent clade~\refFig{tree}, allowing us to make a conclusion about a single \gls{mdrkp} infection source.
All the 5 isolates were also positive for the genes of siderophores yersiniabactin (\textit{ybt}),
aerobactin (\textit{iuc}) and salmochelin (\textit{iro}), genotoxin colibactin (\textit{clb}),
hypermucoidy determinants (\textit{rmpA}, \textit{rmpA2}), specific \textit{wzi} loci alleles related to the
K- (capsule) and O- (LPS) antigens development which cumulative presence corresponds to extremely virulent phenotype
in theory.
More important, their \gls{mlst} was only ST23.
First reported in the mid-1980s in Taiwan, ST23 was often mentioned to present and strongly associated with the
K1 capsular serotype~\cite{Shon2013}.
The \gls{kpne} ST23 strains were confirmed as multidrug-resistant hypervirulent pathogens
caused abscesses of kidneys, pancreas, liver~\cite{Shen2019}, endogenous endophthalmitis~\cite{Xu2018},
with a dominance in the Asian-Pacific region~\cite{Thiry2019}.
In fact, it means the 4 cases of asymptomatic colonization of infant intestinal tract by hypervirulent strains of
\gls{mdrkp} which were spread a short time ago and may be accounted as a hospital-acquired infection.
The persistence of the \gls{mdrkp} ST23 strain has been also confirmed for the patient \#1 (sample \#90)
at 3rd month of life~\refTab{phenotype}~--- still without acute immune response.

The fetal immune system is not developed compared with later life mostly due to the environmental limitation~---
the stress of the own remodelling tissues and the non-inherited maternal alloantigens should not provoke
the immune reaction from the fetus side~\cite{Simon2015}.
Their innate immune system is muted~\cite{Haase2010}, the humoral immune responses are blunted,
the immunoglobulin class switching is incomplete, the released IgG antibodies decline rapidly after
immunization~\cite{Pihlgren2006}.
However, the immature Th1-type T-cell response is compensated by the IFN-$\gamma$ producing
$\gamma\delta$-T-cells~\cite{Gibbons2009}.
By the way, the main mechanism of neonatal immune protection is based on vertical transport of antibodies
via breastfeeding.
With some fortune, breastfed newborns will not suffer infections that have induced the immune response
from their mothers earlier.
Otherwise, a lack of specific antibodies will result an infection development~\cite{Dias2017}.
